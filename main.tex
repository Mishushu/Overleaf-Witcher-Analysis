\documentclass{article}

% Language setting
% Replace `english' with e.g. `spanish' to change the document language
\usepackage[english]{babel}

% Set page size and margins
% Replace `letterpaper' with `a4paper' for UK/EU standard size
\usepackage[letterpaper,top=2cm,bottom=2cm,left=3cm,right=3cm,marginparwidth=1.75cm]{geometry}

% Useful packages
\usepackage{amsmath}
\usepackage{graphicx}
\usepackage[colorlinks=true, allcolors=blue]{hyperref}

\title{Witcher}
\author{Katarzyna Bednarz}

\begin{document}
\maketitle

\begingroup
\centering 
\LaTeX{The story of a monster slayer}

\endgroup

\begin{abstract}
I have a deep affection for the Witcher saga, written by Andrzej Sapkowski. This remarkable series portrays contemporary issues in a fantasy world. Moreover, it captivates readers with its intricately crafted characters and engaging narratives, making it a truly fascinating and enjoyable read.
\end{abstract}

\section{Introduction}


The book delves into the life of the Witcher named Geralt, who perceives himself as a solitary figure during the day. However, despite his self-perception, he inadvertently draws people towards him. His intelligence and wit serve as a magnetic force. Nevertheless, he occasionally exhibits bitterness and pessimism due to his past experiences. While he is engaged in the daily pursuit of hunting monsters for a fee, he holds the belief that the true monsters in the world are, in fact, human beings.


\section{Narrative Overview}

\subsection{How it all started}

Andrzej Sapkowski, the author of the Witcher saga and short stories, reveals that his motivation to write and publish the book stemmed from his son. The initial publications appeared as individual stories in the monthly Fantastyka magazine in 1986. The Witcher garnered substantial enthusiasm and interest from readers following regular releases. As a result, two volumes of short stories were created, paving the way for subsequent volumes in the Witcher saga.

I first encountered the book when a friend lent me the stories to read. Captivated by them, I eagerly sought out the next volumes. My aunt, an avid fantasy fan and collector of Fantastyka magazine, generously made her collectible foiled copies available for me to read. From that moment on, my connection with the series was unwavering. I acquired my own books and regularly revisited them, deriving great joy from each story. My love for The Witcher led me to replay each of the computer games released by CD Projekt Red studio several times. In the following years, I personally collected the first edition of The Witcher and issues of Fantastyka, in which single stories were published. I also shared this passion with my sister, and this multifaceted story firmly took root in my heart.

\subsection{Wandering Monster Slayer}

\begin{figure}
\centering
\includegraphics[width=0.25\linewidth]{1st-edition.jpg}
\caption{\label{fig:geralt}First edition of the Witcher books}
\end{figure}

The life of a Witcher involves constant wandering and risking one's life for a meager reward. During challenging times, people tend to be mistrustful and hesitant to pay for the services of a professional. There are instances when they not only refuse payment but also attempt to rob or negotiate the Witcher's life. Geralt, a Witcher with modest expectations, lives day by day, until he encounters individuals who draw him into various intrigues, romances, and political conflicts.

Despite undergoing rigorous tests in childhood that should theoretically strip him of emotions, Geralt harbors a smoldering spark within, enabling him to follow his heart. His path crosses with the sorceress Jennefer, leading to a complex relationship marked by intense passion, anger, sadness, and longing. Cirilla becomes like a daughter to him, while the renowned bard Jaskier, the dwarf Zoltan, and the vampire Regis become akin to brothers.

As Geralt becomes entangled in political intrigues through a series of events, he is compelled to save his loved ones. In doing so, he makes decisions that have far-reaching consequences for the entire world.


\subsection{Characters}

The characters in the Witcher book series are intricately complex, each harboring stories that significantly impact the narrative. From Geralt, the stoic monster hunter, to Yennefer, the powerful sorceress with a complex relationship, and Ciri, the princess with a mysterious destiny—each character weaves a tale that profoundly influences the life of the Witcher. Their stories, laden with depth and change, contribute to the rich tapestry of the fantasy world crafted by the author.

\begin{table}
\centering
\begin{tabular}{l|r}
Name & Description \\\hline
Jennefer & A powerful sorceress and love interest of Geralt \\
Ciri & A princess with a mysterious destiny \\
Jaskier & A charismatic and witty bard \\
Regis & A vampire, known for his philosophical outlook on life \\
Triss & Another sorceress, plays a significant role in the series \\
Zoltan & A loyal friend. Helpful with a fight and a bottle \\
\end{tabular}
\caption{\label{tab:widgets}Characters}
\end{table}

\subsection{Books}

\begin{table}
\centering
\begin{tabular}{l|r}
Title & No of Pages \\\hline
1.The Last Wish & 292 \\
8.Season of storms & 404 \\
2.Sword of Destiny & 384 \\
3.Blood of Elves & 416 \\
4.Time of Contempt & 352  \\
5.Baptism of Fire & 464 \\
6.The Tower of the Swallow & 464 \\
7.The Lady of the Lake & 531 \\

\end{tabular}
\caption{\label{tab:widgets}The Witcher stories and saga}
\end{table}

\subsection{How to add Citations and a References List}

\say{Here, a quotation is written and even some \say{This World Doesn't Need A Hero; It Needs A Professional.} quotations 
are possible}

\subsection{Posdumowanie}

\bibliographystyle{alpha}
\bibliography{sample}
\url{https://lubimyczytac.pl/jak-czytac-wiedzmina-kolejnosc-czytania-sagi-sapkowskiego}.

\end{document}